\chapter*{}
%\thispagestyle{empty}
%\cleardoublepage

%\thispagestyle{empty}

\input{portada/portada_2}



\cleardoublepage
\thispagestyle{empty}

\begin{center}
{\large\bfseries Adoptapp: Proyecto para la adopción y acogida de animales}\\
\end{center}
\begin{center}
Jordi Pereira Gil (alumno)\\
\end{center}

%\vspace{0.7cm}
\noindent{\textbf{Palabras clave}: animal, adopción, acogida, voluntariado, asociación, aplicación}\\

\vspace{0.7cm}
\noindent{\textbf{Resumen}}\\

En la actualidad, si una persona está interesada en adoptar un animal, quiere ser voluntaria en una asociación, o desea informar que se ha perdido o encontrado un animal, no dispone de ninguna aplicación para hacer estas acciones. Tendría que buscar una asociación y dirigirse personalmente a ella, o bien anunciarse en redes sociales o poner letreros en la calle o en veterinarias indicando qué desea (adoptar, buscar, acoger, etc.). Todas estas son acciones que requieren bastante esfuerzo y una alta motivación. No hay una centralización de información, y la difusión por redes sociales suele ser en un formato no persistente y poco organizado, en especial para las asociaciones de animales.

Para paliar estos problemas, hemos decidido crear una aplicación multiplataforma que proporcione a asociaciones de animales y ciudadanos un entorno en el que poder agilizar las adopciones y acogidas de animales, además de poder publicar alertas de animales perdidos entre otras funcionalidades. El objetivo es facilitar la comunicación entre las dos entidades previamente mencionadas.

En cuanto al desarrollo, hemos utilizado la metodología ágil \textit{Scrum} con 4 iteraciones de una duración aproximada de un mes cada una. Esto nos ha permitido programar gran parte de lo deseado, no obstante, no hemos podido cumplir con todas las tareas debido a la falta de tiempo.
\cleardoublepage


\thispagestyle{empty}


\begin{center}
{\large\bfseries \myTitle: Project for the adoption and fostering of animals}\\
\end{center}
\begin{center}
Jordi, Pereira Gil (student)\\
\end{center}

%\vspace{0.7cm}
\noindent{\textbf{Keywords}: animal, adoption, fostering, volunteering, association, application}\\

\vspace{0.7cm}
\noindent{\textbf{Abstract}}\\

Nowadays , if a person is interested in adopting an animal, wants to volunteer in an association, or wants to report that an animal has been lost or found, there is no application available to do this. They would have to look for an association and contact them personally, or advertise on social networks or put up signs on the street or in veterinary surgeries indicating what they want (adopt, search, foster, etc.). These are all actions that require a lot of effort and a high level of motivation. There is no centralisation of information, and dissemination through social networks is often in a non-persistent and unorganised format, especially for animal associations.

To alleviate these problems, we have decided to create a multi-platform application that provides animal associations and citizens with an environment in which they can speed up animal adoptions and fostering, as well as publish alerts for lost animals, among other functionalities. The aim is to facilitate communication between the two aforementioned entities.

As for the development, we have used the agile methodology \textit{Scrum} with 4 iterations of a duration of approximately one month each. This has allowed us to programme a large part of what we wanted, however, we have not been able to complete all the tasks due to lack of time.

\chapter*{}
\thispagestyle{empty}

\noindent\rule[-1ex]{\textwidth}{2pt}\\[4.5ex]

Yo, \textbf{Jordi Pereira Gil}, alumno de la titulación GRADO EN INGENIERÍA INFORMÁTICA de la \textbf{Escuela Técnica Superior
de Ingenierías Informática y de Telecomunicación de la Universidad de Granada}, con DNI 35674006V, autorizo la
ubicación de la siguiente copia de mi Trabajo Fin de Grado en la biblioteca del centro para que pueda ser
consultada por las personas que lo deseen.

\vspace{6cm}

\noindent Fdo: Jordi Pereira Gil

\vspace{2cm}

\begin{flushright}
Granada a 1 de junio de 2025.
\end{flushright}


\chapter*{}
\thispagestyle{empty}

\noindent\rule[-1ex]{\textwidth}{2pt}\\[4.5ex]

D. \textbf{	María José Rodríguez Fórtiz (tutor1)}, Profesora del Área de Lenguajes y Sistemas Informáticos del Departamento Lenguajes y Sistemas Informáticos de la Universidad de Granada.

\vspace{0.5cm}


\vspace{0.5cm}

\textbf{Informan:}

\vspace{0.5cm}

Que el presente trabajo, titulado \textit{\textbf{\myTitle, \mySubTitle}},
ha sido realizado bajo su supervisión por \textbf{\myName}, y autorizamos la defensa de dicho trabajo ante el tribunal
que corresponda.

\vspace{0.5cm}

Y para que conste, expiden y firman el presente informe en Granada a X de junio de 2025 .

\vspace{1cm}

\textbf{Los directores:} 
	María José Rodríguez Fórtiz

\vspace{5cm}

\noindent \textbf{\myProf}

\chapter*{Agradecimientos}
\thispagestyle{empty}

       \vspace{1cm}


A mi madre, que me ha permitido estudiar una carrera tan lejos, a pesar de que no haber sido fácil.

A mi tutora, que siempre me ha orientado cuando lo he necesitado y me ha ayudado enormemente.

