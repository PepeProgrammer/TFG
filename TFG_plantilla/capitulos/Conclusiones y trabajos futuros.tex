\chapter{Conclusiones y trabajos futuros}

\section{Conclusiones}

La conclusión fundamental de este trabajo en cuanto al objetivo principal: desarrollar una aplicación móvil multiplataforma para ciudadanos y asociaciones o protectoras de animales que facilite la adopción y acogida, así como la comunicación entre estos, es que se ha realizado casi por completo, hemos conseguido desarrollar una aplicación que facilite el proceso de acogida y adopción para todas las partes, como podemos ver en el apartado \ref{propuesta}. No obstante, por falta de tiempo no se ha llegado a desarrollar el sistema de comunicación entre asociaciones y ciudadanos. A pesar de esto, consideramos que los resultados han sido positivos, ya que hemos llevado a cabo satisfactoriamente todo el desarrollo que hemos podido, adaptándonos adecuadamente cada vez que surgía un contratiempo.

Como podemos apreciar en el apartado \ref{rev}, hemos analizado diferentes tecnologías para decidir cuales eran las más adecuadas para el proyecto. Consideramos que debido a los conocimientos previos poseídos y las opciones disponibles, hemos hecho una elección bastante acertada, ya que no hemos empezado totalmente de 0 y eso nos ha permitido empezar con el desarrollo lo antes posible.

No obstante, si que ha habido un proceso de aprendizaje previo importante, ya que, aunque tanto en el servidor como en el cliente utilizamos Javascript, hemos tenido que adaptarnos a diferentes entornos de desarrollo, con sus propios elementos y formas de trabajar. A lo largo del desarrollo se ha notado la mejora y familiarización con las tecnologías, pudiendo afirmar que se han aprendido de manera satisfactoria.
 
Por otra parte, hemos revisado varias aplicaciones similares en el apartado \ref{appssimilares} para analizar sus carencias a nivel funcional. A partir de este análisis, hemos obtenido información que nos ha permitido realizar mejoras en la planificación previa al desarrollo, permitiéndonos también, saber que debemos hacer para que la aplicación resultante sea útil y llamativa para los usuarios.

Consideramos también que la aplicación mejora de manera substancial la forma en la que se gestionan las adopciones y acogidas con respecto a las formas convencionales de difusión mediante redes sociales. Esto se debe, a que hemos creado una plataforma a la que cualquier asociación puede sumarse, permitiendo a los ciudadanos tener que acceder a un única plataforma para obtener información sobre que animales hay en adopción, acogida y/o animales perdidos y encontrados. Evidentemente esto dependerá de la popularidad de la aplicación.

A nivel personal, gracias a colaborar con la asociación Colonias Felinas Armilla, teniendo un par de reuniones previas a la planificación de las historias de usuario, como podemos ver en el apartado \ref{it0}, siento que me ha ayudado a entender mejor como funcionan los flujos de trabajo fuera del ámbito académico, acercándome más al mundo laboral, teniendo que desarrollar habilidades de comunicación que son difíciles de conseguir de otra manera.

Por último, gracias a haber elegido una metodología ágil para el desarrollo de la aplicación, considero que mis conocimientos de la ingeniería del software se han ampliado, ya que he tenido que trabajar con herramientas específicas que me han ayudado a entender mejor el desarrollo del software.  


	
\section{Trabajos futuros}

A pesar de que el desarrollo ha ido bien, por falta de tiempo no hemos podido hacer todas las cosas que están pensadas para la aplicación. No hemos podido implementar el sistema de mensajería para hacer más eficiente la comunicación entre las asociaciones y los ciudadanos, tampoco hemos podido añadir el sistema de voluntariado, donde los usuarios particulares podrían aceptar ser voluntarios en distintas asociaciones.

Además, no se han hecho tareas relacionadas con los administradores ni los moderadores, ya que queríamos priorizar tener la máxima cantidad de páginas para las asociaciones y los usuarios particulares debido a que es lo que consideramos más importante en la aplicación. Todas estas tareas quedarían para futuras iteraciones.

Por otra parte, no hemos podido crear un entorno de producción, debido a que no se ha encontrado ninguna alternativa para almacenar el servidor gratuita. En el apartado \ref{presupuesto}, el del presupuesto, hemos dicho que vamos a usar \textit{amazon aws}, pero al ser una herramienta de pago no se han podido realizar pruebas.



