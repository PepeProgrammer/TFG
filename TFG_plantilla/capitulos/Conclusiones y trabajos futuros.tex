\chapter{Conclusiones y trabajos futuros}

\section{Conclusiones}

En este TFG se ha trabajado con mucho interés y de forma constante para cumplir con todos los objetivos planteados, aunque en el desarrollo se han implementado solo las funcionalidades más prioritarias por limitaciones de tiempo, quedando otras como trabajos futuros. A continuación, revisaremos los objetivos enumerados en el apartado \ref{objetivos} y explicaremos cómo se han alcanzado:

\begin{itemize}
	\item Desarrollar una aplicación móvil multiplataforma para ciudadanos y asociaciones o protectoras de animales que facilite la adopción y acogida, así como la comunicación entre estos: Este objetivo consideramos que se ha realizado casi por completo, hemos conseguido desarrollar una aplicación que facilite el proceso de acogida y adopción para todas las partes, como podemos ver en el apartado \ref{propuesta}. No obstante, por falta de tiempo no se ha llegado a desarrollar el sistema de comunicación entre asociaciones y ciudadanos. A pesar de esto, consideramos que los resultados han sido positivos, ya que hemos llevado a cabo satisfactoriamente todo el desarrollo que hemos podido, adaptándonos adecuadamente cada vez que surgía un contratiempo.
	
	\item Analizar diferentes tecnologías candidatas para ser usadas en el desarrollo: Como podemos apreciar en el apartado \ref{rev}, hemos analizado diferentes tecnologías para decidir cuales eran las más adecuadas para el proyecto. Consideramos que debido a los conocimientos previos poseídos y las opciones disponibles, hemos hecho una elección bastante acertada, ya que no hemos empezado totalmente de 0 y eso nos ha permitido empezar con el desarrollo lo antes posible.
	
	\item Profundizar en el aprendizaje de las tecnologías seleccionadas para ser más eficiente y mejorar mi formación: A pesar de lo mencionado en el anterior objetivo, si que ha habido un proceso de aprendizaje previo importante, ya que, aunque tanto en el servidor como en el cliente utilizamos TypeScript, hemos tenido que adaptarnos a diferentes entornos de desarrollo, con sus propios elementos y formas de trabajar. A lo largo del desarrollo se ha notado la mejora y familiarización con las tecnologías, pudiendo afirmar que se han aprendido de manera satisfactoria.
	
	\item Revisar aplicaciones similares para analizar carencias a nivel funcional: Este objetivo está reflejado en el apartado \ref{appssimilares}. A partir de este análisis, hemos obtenido información que nos ha permitido realizar mejoras en la planificación previa al desarrollo, permitiéndonos también, saber que debemos hacer para que la aplicación resultante sea útil y llamativa para los usuarios.
	
	\item Proporcionar un apoyo software para aumentar la eficiencia de las adopciones y acogidas de los animales: Para cumplir este objetivo hemos buscado que haya facilidades de búsqueda para los ciudadanos a la hora de buscar animales en la zona. Para ello hemos creado páginas que facilitan dicha búsqueda mediante el uso de filtros, evitando que tengan que buscar la información de la manera convencional, a través de las redes sociales de las asociaciones.
	
	\item Crear una plataforma con comunidad para hacer más rápido el encuentro de animales perdidos: Este objetivo está estrechamente relacionado con el anterior, ya que tanto como para acelerar el encuentro de animales perdidos como para mejorar la eficiencia de las adopciones necesitamos que la gente utilice la aplicación. Con esto en mente hemos desarrollado una aplicación intuitiva que resulte familiar al mayor número de usuarios posibles.
	
	\item Proporcionar un apoyo de software para facilitar el trabajo a las protectoras y asociaciones: Después de todo el desarrollo, consideramos que hemos creado una aplicación que cumple con esta función, ya que tienen un registro digitalizado de sus animales, con datos específicos de cada uno. Debido a la priorización de tareas de mayor relevancia no es todo lo completo que nos gustaría, no obstante en iteraciones futuras se realizarían mejoras en este apartado.
	
	\item Centralizar la información de las asociaciones en un espacio común para facilitar su búsqueda a los usuarios adoptantes: Hemos creado una plataforma a la que cualquier asociación puede sumarse, permitiendo a los ciudadanos tener que acceder a un única plataforma para obtener información sobre que animales hay en adopción y acogida.
	
	\item Realizar una aplicación en colaboración con una entidad real, una asociación de animales, para mejorar mis habilidades blandas, específicamente aprender a comunicarme con ella y a cubrir los requisitos consensuados: A nivel personal, gracias a colaborar con la asociación Colonias Felinas Armilla, teniendo un par de reuniones previas a la planificación de las historias de usuario, como podemos ver en el apartado \ref{it0}, siento que me ha ayudado a entender mejor como funcionan los flujos de trabajo fuera del ámbito académico, acercándome más al mundo laboral, teniendo que desarrollar habilidades de comunicación que son difíciles de conseguir de otra manera.
	
	\item Mejorar mis conocimientos en ingeniería del software al aplicar metodologías ágiles de desarrollo y usar herramientas de especificación y diseño: Gracias a haber elegido una metodología ágil para el desarrollo de la aplicación, considero que mis conocimientos de la ingeniería del software se han ampliado, ya que he tenido que trabajar con herramientas específicas que me han ayudado a entender mejor el desarrollo del software.
	\end{itemize}

  Esta aplicación se alinea con varios Objetivos de Desarrollo Sostenible (ODS) \cite{ods} de la Agenda 2030 de la ONU. En primer lugar, contribuye al ODS 15 (Vida de ecosistemas terrestres) al promover el bienestar animal y reducir el abandono, fomentando una tenencia responsable. Además, al digitalizar y centralizar información para asociaciones y ciudadanos, impulsa el ODS 9 (Industria, innovación e infraestructura), ya que utiliza tecnología accesible para optimizar procesos. También conecta con el ODS 11 (Ciudades y comunidades sostenibles), al crear una plataforma colaborativa que fortalece redes locales de protección animal. Por último, al priorizar la comunicación y el trabajo en equipo con entidades reales, refleja el ODS 17 (Alianzas para lograr los objetivos), demostrando cómo la cooperación entre desarrolladores y organizaciones puede generar soluciones con impacto social y ambiental.
  
  La colaboración con la asociación Colonias Felinas Armilla, junto con la supervisión de mi tutora y la necesidad de ajustarme a requisitos reales, ha fortalecido significativamente mis habilidades blandas. Trabajar con una entidad externa me obligó a mejorar mi comunicación, aprendiendo a escuchar necesidades concretas, traducirlas en requisitos técnicos y presentar avances de forma clara y profesional. Además, al recibir feedback, desarrollé capacidad de adaptación, ajustando el proyecto cuando surgían cambios o prioridades distintas a las iniciales.
  
  Esta experiencia también me ha enseñado la importancia de la gestión de expectativas, ya que en un proyecto real no siempre es posible implementar todas las funcionalidades ideales. Por último, al aplicar metodologías ágiles, he mejorado mi capacidad de organización, habilidad clave para cualquier entorno laboral. En definitiva, la realización de este TFG me ha permitido acercarme al mundo real, donde debemos tener un mejor balance entre las tareas a realizar y el tiempo del que disponemos.
  
  Durante el desarrollo del proyecto he notado una gran mejoría en mis habilidades de programación además de las mencionadas previamente. Siento que he sido capaz de adaptarme bien a los cambios que han ido surgiendo y me ha dado confianza el hecho de ver que soy solvente y resolutivo en proyectos reales, lo que sé que mejorará mi entrada al mundo laboral, ya que me ha permitido desarrollar habilidades que en el contexto académico son difíciles de obtener, debido a que los desarrollos suelen ser muy cortos y generalmente se da feedback al final del mismo.


	
\section{Trabajos futuros}

A pesar de que el desarrollo ha ido bien, había un gran número de requisitos que abordar y por ello se han priorizado para que diera tiempo a hacer los más importantes. Hemos dejado para trabajos futuros el sistema de mensajería para hacer más eficiente la comunicación entre las asociaciones y los ciudadanos, éste se realizará al estilo de las aplicaciones de mensajería tradicionales y habrá que crear una tabla que almacene los mensajes entre los distintos actores.

Otra tarea que será un trabajo futuro es la añadir el sistema de voluntariado, donde los usuarios particulares podrían aceptar ser voluntarios en distintas asociaciones. Para ello debemos crear una página en la que los ciudadanos puedan aceptar solicitudes de voluntariado por parte de las asociaciones y almacenar la información de las fechas y el tiempo en las que se realiza el voluntariado, para que las asociaciones tengan a su disposición dicha información, ayudándolas con la organización de las mismas. 

Además, realizaremos las tareas relacionadas con los administradores y los moderadores, ya que queríamos priorizar tener la máxima cantidad de páginas para las asociaciones y los usuarios particulares debido a que es lo que consideramos más importante en la aplicación. Los administradores tendrán capacidades básicas como la creación, modificación y eliminación de usuarios así como la capacidad de crear moderadores. 

Los moderadores, por otra parte se encargarán de que los datos subidos por asociaciones y ciudadanos sean consistentes, pudiendo eliminar las publicaciones o cuentas que consideren que no estén alineadas con los contenidos deseados en la aplicación.

Por último, crearemos un entorno de producción, debido a que no se ha encontrado ninguna alternativa para almacenar el servidor de forma gratuita. En el apartado \ref{presupuesto}, el de costes, hemos dicho que vamos a usar \textit{amazon aws}, por lo que será el entorno que utilizaremos para almacenar el servidor de nuestra aplicación.



