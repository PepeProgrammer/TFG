
\chapter{Introducción}

Este trabajo tratará sobre el desarrollo de una aplicación para la gestión de las asociaciones de animales en diferentes acciones como por ejemplo las adopciones y acogidas.

\section{Motivación}

El desarrollo de este proyecto nace debido a la falta de organización con respecto a los avisos de adopciones ya la falta de centralización de dicha información, ya que la mayoría de las veces se difunden por las distintas redes sociales pero suele ser en un formato no persistente y poco organizado, en especial para las asociaciones de animales.

Esta idea se llevaba macerando varios años, ya que siempre que aparecía cualquier tipo de información sobre animales perdidos era en un formato temporal y solo visible en función de un algoritmo basado en tus preferencias personales, por lo que siempre aparecía el pensamiento de "ojalá esto llegara a la gente sí o sí". Básicamente ese fue el germen de AdoptApp, ya que aunque empezó por las publicaciones de animales perdidos, eso ha llevado a la idea de que también sería muy positivo que se pudiera ver que animales hay en adopción o para acogida en la zona.

Además, también nos hemos decantado por esta propuesta debido a que las asociaciones y protectoras realizan un trabajo enorme, sin ánimo de lucro, por lo que consideramos pertinente realizar un sistema con el que pudiesen tener las mayores facilidades posibles para dicho trabajo.

\section{Objetivos}

\begin{itemize}
	\item Crear una aplicación para la adopción y acogida de animales. %objetivo específico
	\item Revisar aplicaciones similares para analizar carencias a nivel funcional. %: No hay una gran cantidad de aplicaciones de este estilo, pero como veremos en el capítulo 3 en la parte de revisión de aplicaciones similares \ref{appssimilares}, no hay demasiadas.
	\item Proporcionar un apoyo software para aumentar la eficiencia de las adopciones y acogidas de los animales. %: con AdoptApp buscamos principalmente que se consigan adoptar y/o acoger a la mayor cantidad de animales posible, por lo que trataremos de crear una aplicación que permita dar visibilidad y agilidad a ese proceso.
	\item Proporcionar un apoyo de software para facilitar el trabajo a las protectoras y asociaciones.
	\item Centralizar la información de las asociaciones en un espacio común para facilitar su búsqueda a los usuarios adoptantes.
	\item Crear una plataforma con comunidad para hacer más rápido el encuentro de animales perdidos.
	
\end{itemize}

\section{Planificación temporal}

\section{Presupuesto}

Para el entendimiento del presupuesto debemos tener en cuenta las siguientes consideraciones:

\begin{itemize}
	\item Para la realización de dicho presupuesto suponemos el desarrollo completo del sistema lo que se estima que serán unas 600 horas totales.
	\item Supondremos que únicamente hay un desarrollador a cargo del proyecto.
	\item Como veremos en el apartado de \textit{Dominio del problema} \ref{dompro}, donde explicaremos con más detalle cada apartado del que vemos en el presupuesto, vamos a usar \textit{amazon aws} para el servidor, por lo que los precios serán los que define amazon \cite{aws}
	\item Para el calculo de la amortización de los equipos informáticos, básicamente hemos dividido el número de horas entre 8, suponiendo que el ritmo de trabajo es de 8 horas diarias 5 días a la semana, y ese mismo número dividiéndolo entre 5 para obtener el número de semanas que se va a utilizar el ordenador teóricamente. Con todo esto obtenemos la cifra que aparece en el presupuesto.
	\item El cálculo de los impuestos de la seguridad social se ha calculado en base a lo que cobra en bruto un informático en España y sabiendo cuanto es el sueldo neto medio \cite{jobted}. Según la página de \textit{Jobted} el salario bruto medio es de 30300€ al año, mientras que el salario mensual neto es de 1640€. Si dividimos la cantidad bruta entre 14, el número de pagas anuales, obtenemos una cantidad 2164,29€ aproximadamente. Al realizar la resta de éste entre el salario neto obtenemos una cantidad de 524,28€ euros al mes en impuestos. 
	
	Con la información del sueldo en bruto mensual podemos calcular el precio hora que cuesta un trabajador, unos 13,52€/hora, pero para redondear, en el presupuesto hemos puesto 14€/hora.
	
\end{itemize}

Todo esto es susceptible a modificaciones en función de los cambios que se vayan produciendo en el proyecto, así como en el caso de que haya algún retraso en la programación, pero el presupuesto estimado quedaría de la siguiente manera:

\begin{table}[h]
	\centering
	\begin{tabular}{lr}
		\toprule
		\textbf{Concepto} & \textbf{Coste (EUR)} \\
		\midrule
		\textbf{Desarrollo de software (600 horas)} & \\
		\quad Desarrollo frontend (250 horas × 14€/h) & 3500€ \\
		\quad Desarrollo backend (350 horas × 14€/h) & 4900€ \\
		\midrule
		\textbf{Infraestructura AWS (mensual)} & \\
		\quad EC2 t3.medium (servidor) & 35€ \\
		\quad RDS db.t3.micro (base de datos) & 15€ \\
		\quad S3 (almacenamiento) & 5€ \\
		\quad CloudFront (CDN) & 10€ \\
		\quad Route 53 (DNS) & 0,50€ \\
		\quad \textbf{Total mensual AWS} & \textbf{65,50€} \\
		\midrule
		\textbf{Costes adicionales} & \\
		\quad Dominio (.app o .com) & 15€/año \\
		\quad Certificado SSL & 0€ (usando ACM) \\
		\midrule
		\textbf{Coste de adquisición de material inventariable} \\
		\quad Amortización de equipos informáticos & 38,46€ \\
		\midrule
		\textbf{Total coste inicial (desarrollo + primer mes)} & \textbf{8453,46€} \\
		\textbf{Total coste recurrente (anual)} & \textbf{15€} \\
		\textbf{Total coste recurrente (mensual)} & \textbf{65.50€} \\
		\bottomrule
	\end{tabular}
		\caption{Desglose de costes del proyecto}
\end{table}
