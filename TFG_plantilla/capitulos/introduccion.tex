
\chapter{Introducción}

Este trabajo tratará sobre el desarrollo de una aplicación para la gestión de las asociaciones de animales en diferentes acciones como por ejemplo las adopciones y acogidas.

\section{Motivación}

El desarrollo de este proyecto nace debido a la falta de organización con respecto a los avisos de adopciones ya la falta de centralización de dicha información, ya que la mayoría de las veces se difunden por las distintas redes sociales pero suele ser en un formato no persistente y poco organizado, en especial para las asociaciones de animales.

Esta idea se llevaba macerando varios años, ya que siempre que aparecía cualquier tipo de información sobre animales perdidos era en un formato temporal y solo visible en función de un algoritmo basado en tus preferencias personales, por lo que siempre aparecía el pensamiento de "ojalá esto llegara a la gente sí o sí". Básicamente ese fue el germen de AdoptApp, ya que aunque empezó por las publicaciones de animales perdidos, eso ha llevado a la idea de que también sería muy positivo que se pudiera ver que animales hay en adopción o para acogida en la zona.

Además, también nos hemos decantado por esta propuesta debido a que las asociaciones y protectoras realizan un trabajo enorme, sin ánimo de lucro, por lo que consideramos pertinente realizar un sistema con el que pudiesen tener las mayores facilidades posibles para dicho trabajo.

\section{Objetivos}

\begin{itemize}
	\item Crear una aplicación para la adopción y acogida de animales. %objetivo específico
	\item Revisar aplicaciones similares para analizar carencias a nivel funcional. %: No hay una gran cantidad de aplicaciones de este estilo, pero como veremos en el capítulo 3 en la parte de revisión de aplicaciones similares \ref{appssimilares}, no hay demasiadas.
	\item Proporcionar un apoyo software para aumentar la eficiencia de las adopciones y acogidas de los animales. %: con AdoptApp buscamos principalmente que se consigan adoptar y/o acoger a la mayor cantidad de animales posible, por lo que trataremos de crear una aplicación que permita dar visibilidad y agilidad a ese proceso.
	\item Proporcionar un apoyo de software para facilitar el trabajo a las protectoras y asociaciones.
	\item Centralizar la información de las asociaciones en un espacio común para facilitar su búsqueda a los usuarios adoptantes.
	\item Crear una plataforma con comunidad para hacer más rápido el encuentro de animales perdidos.
	
\end{itemize}

\section{Planificación temporal}

\section{Presupuesto}
