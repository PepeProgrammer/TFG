\section{Estado del Arte}

\subsection{Dominio del problema}

\subsection{Revisión de tecnologías: alternativas}
En esta sección vamos a comparar las distintas tecnologías para el desarrollo de aplicaciones híbridas y entre todas las vistas escogeremos las que sean más optimas para el desarrollo de dicha aplicación.

\subsubsection{IDEs}
\subsubsection{Control de versiones}
\subsubsection{Front-end}
Primero hablaremos de las tecnologías para el front, para ello se muestran las siguientes alternativas.

\begin{table}[H] %Esto hace que se ponga donde le corresponde en lugar de "flotar"
    \centering
    \begin{tabular}{|p{2cm} |p{4 cm} |p{4cm} |p{4cm} |} \hline 
         &  \textbf{Flutter}&  \textbf{Ionic}& \textbf{React Native}\\  \hline 
         Descripción &  Marco multiplataforma de código abierto de Google para desarrollar aplicaciones iOS/Android &  Marco frontend para crear aplicaciones móviles para teléfonos iOS/Android que utilizan la misma base de código& Framework de programación de aplicaciones nativas multiplataforma, para desarrollar apps iOS/Android\\ \hline 
         
        Lenguaje de programación &  Dart&  HTML,CSS y Javascript & Javascript\\ \hline 
        API nativas &  Si&  Si & Si\\ \hline 
        Despliegue &  Móvil, Web, Escritorio&  Móvil, Web, PWA, Escritorio & Móvil\\ \hline 
        Rendimiento móvil &  Excelente &  Bueno & Muy bueno\\ \hline 
        Rendimiento web &  Deficiente &  Excelente & No tiene\\ \hline 
    \end{tabular}
    \caption{Comparativa tecnologías front-end}
    \label{tab:tec_front}
\end{table}

Bibliografía \\

https://blog.back4app.com/es/flutter-vs-ionic/ \\
https://www.thedroidsonroids.com/blog/flutter-vs-react-native-comparison

\subsubsection{Back-end}

En cuanto al back, debemos separar entre las tecnologías de almacenamiento de datos y tecnologías de obtención y recepción de datos del front-end. \\

\textbf{Bases de datos} \\ \\
En esta sección vamos a coger una de las bases de datos relacionales más populares con la base de datos no relacional más popular. 
\begin{table}[H] %Esto hace que se ponga donde le corresponde en lugar de "flotar"
    \centering
    \begin{tabular}{|p{2cm} |p{4 cm} |p{4cm} |} \hline 
         &  \textbf{MongoDb}&  \textbf{MySQL}\\  \hline 
         Tipo &  NoSQL &  SQL \\ \hline 
         
        Estructura de la base de datos &  Almacena los datos en documentos y colecciones de tipo JSON &  Almacena los datos en una estructura tabular con filas y columnas \\ \hline 
        Arquitectura &  Nexus &  Cliente servidor\\ \hline 
        Lenguaje de consulta &  MQL &  SQL\\ \hline 
        Rendimiento &  Es más rápido que MySQL y facilita las peticiones de lectura y escritura rápidas &  	Es relativamente más lento que MongoDB al manejar grandes volúmenes de datos, ya que almacena los datos en formato tabular \\ \hline 
        Seguridad &  Al no tener una estructura fija, pueden surgir incoherencias y problemas de seguridad de los datos &  ofrece una mayor seguridad ya que tiene estructuras de datos definidas con mayor consistencia \\ \hline
        Escalabilidad &  Es altamente escalable y ofrece escalado horizontal a través de sharding &  Su escalabilidad es limitada, y tienes la opción de escalar mediante réplicas de lectura o escalado vertical \\ \hline
        Complejidad &  Fácil &  Muy fácil \\ \hline
        
    \end{tabular}
    \caption{Comparativa tecnologías BBDD}
    \label{tab:tec_db}
\end{table}

https://kinsta.com/es/blog/mongodb-vs-mysql/ \\ \\

\textbf{Lenguajes de programación back} \\ \\
\begin{table}[H] %Esto hace que se ponga donde le corresponde en lugar de "flotar"
    \centering
    \begin{tabular}{|p{2cm} |p{4 cm} |p{4cm} |p{4cm} |} \hline 
         &  \textbf{Node}&  \textbf{Python}& \textbf{PHP}\\  \hline 
         Freamwork &  Express &  Django & Laravel\\ \hline 
         
        Conectividad BBDD &  Dart&  La conectividad de la base de datos es posible, pero no para todos. Además, necesita controladores. & Conectividad sencilla, capaz de conectarse con más de 25 bases de datos sin problemas.\\ \hline 
        Curva de aprendizaje &  Muy fácil si tienes conocimento de javascript& Fácil & Media\\ \hline 
        Seguridad &  ofrece prácticas de seguridad que permiten securizar el sistema comodamente&  Más seguro con funciones de ciberseguridad integradas & Se han producido muchos ataques a la seguridad	\\ \hline 
        velocidad &  Más rápido que php &  Rápido(menos que php) & Muy rápido a partir de la versión 7\\ \hline 
        Popularidad & Un poco más popular que python 1.8\% &  Menos popular que PHP (alrededor del 1,1\% de todos los sitios de Internet utilizan Python) & Más popular (cerca del 79\% de los sitios web utilizan PHP)\\ \hline 
    \end{tabular}
    \caption{Comparativa tecnologías back-end}
    \label{tab:tec_back}
\end{table}



https://keepcoding.io/blog/tecnologias-backend/ \\
https://kinsta.com/es/blog/node-js-vs-php/


\subsection{Revisión de aplicaciones similares}

ASPCA Volunteer Portal: https://play.google.com/store/apps/details?id=org.aspca.volunteer&pli=1
Amazdog: https://amazdog.com